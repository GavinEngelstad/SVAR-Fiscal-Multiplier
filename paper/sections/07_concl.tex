This paper estimates the fiscal multiplier using a structural VAR. We impose a simultaneous relationship between GDP, government spending, and revenue à la \textcite{blanchard2002empirical}. Then, we analyze the effect of a structural government spending shock. Analysis of the GDP response to a spending shock finds a \$1 increase in government spending causes a slightly delayed approximate \$1 increase in GDP. This result is robust to minor specification changes and, ignoring specific years at the end of the 60s and start of the 90s, is relatively consistent over time.

Our analysis relies on key assumptions about the structure of macroeconomic relationships. We assume government spending has a delayed response to changes in GDP, a certain value for the government revenue response to GDP changes, and need to impose additional identification restrictions to uncover the causal effect. Our estimate is robust to minor changes in these assumptions, but we do not test our estimate against specifications that impose a completely different set of structural VAR assumptions. These could include long-run conditions that allow more simultaneous relationships to be estimated, sign restrictions, or Bayesian priors \parencites{mountford2009effects}{afonso2019fiscal}.

There is also evidence that the multiplier effect evolves over time in response to macroeconomic events. Particularly, estimates suggest the size of the multiplier changes throughout the business cycle \parencites{baum2012fiscal}{albonico2021public}. Specifically, spending multipliers are larger during periods of low economic activity \parencite{arin2015fiscal}. This would suggest the volatility seen in Figure \ref{fig:ts} is not noise, and could represent a real change in the multiplier.

Still, we believe our structural VAR approach does reasonably well at estimating the fiscal multiplier. Future work could address this paper's limitations with alternative structural frameworks. It could also analyze heterogeneity in the multiplier, either over time, across different states, or between different countries. Additionally, economic theory suggests the revenue-side multiplier, which this paper ignores, should behave similarly to the spending-side multiplier, although empirical work often finds revenue multipliers are smaller \parencite{mineshima2014size}. Future work could test the effects of structural shocks to government revenue, not just spending.
