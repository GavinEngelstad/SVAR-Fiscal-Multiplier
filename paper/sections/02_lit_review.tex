
Previous studies suggest underestimation of the multiplier led to misestimation of growth during financial crises and that use of ``normal times" multipliers to guide policy during recessions leads to systemic bias \parencites{blanchard2013growth, blanchard2014learning}.  Estimates suggest the size of the multiplier changes with the stage of the business cycle and certain multipliers correlate with steady state values, with the debt multiplier increasing with the steady state debt level \parencites{baum2012fiscal, albonico2021public}.  Similarly, spending and tax multipliers behave differently depending on the level of aggregate economic activity.  Tax multipliers become larger during periods of high activity while spending multipliers are larger during low periods of activity \parencite{arin2015fiscal}.

Measuring the multiplier also depends heavily on the method of estimation used.  Longer time series, breaking Ricardian equivalence for higher numbers of agents, modelling central banks that must decide policy given fiscal policy lead to much higher estimates than most methods while times series that end in recent years lead to lower estimates \parencite{gechert2012fiscal}.  Unanticipated fiscal consolidations can lead to long-term increases in inequality \parencite{furceri2022distributional}
% Implications on income inequality when incorporating expectations. 
%There needs to be more here

A popular method of estimation for multipliers SVAR.  \textcite{sims1980macroeconomics} introduce structural restrictions to vector autoregression (VAR) models which allows for causal interpretation of the estimates.  This general framework has been applied to study postwar business cycle fluctuations, oil shocks, the effect of monetary policy regime on output, and fiscal policy \parencites{hamilton1983oil, hodrick1997postwar, sims2006were,blanchard2002empirical}. 

% Increasing taxes reduces GDP \parencite{barro2011macroeconomic}

% Stage of business cycle changes size of multiplier \parencite{baum2012fiscal}.


Structural parameters for Uhlig's restrictions fail to satisfy restrictions on systematic components of monetary policy \parencite{arias2019systematic}.

% As steady state debt level increases, so too does the debt multiplier \parencite{albonico2021public}.  (Implications for fiscal policy during recessions)

% Spending multiplier larger during periods of low economic activity while tax multipliers larger during periods of high economic activity \parencite{arin2015fiscal}

%SVARS



Multiplier estimated through
\begin{itemize}
    \item using one through seven quarter ahead forecasts of key macroeconomic variables such as consumption, output, and wages to control for agents' information sets \parencite{hall2023economic}
\end{itemize}


% Estimate of multiplier depends heavily on method of estimation \parencite{gechert2012fiscal}.


