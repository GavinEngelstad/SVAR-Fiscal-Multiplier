 %importance of good fiscal policy

 %Regulation of economy is important (political economy stuff??)
Monetary and fiscal policy are the two main channels through which the US economy is regulated.  Monetary policy involves the central bank exercising power over the aggregate money supply to change interest rates in an effort to impact consumption decisions.  Fiscal policy describes the federal government's efforts to affect the economy through the two main mechanisms of public spending and taxes.  
% Competing effects necessitate an optimal policy mix 

%Need some transition to just talking about fiscal

%how the multiplier factors into this

Competing theories of the effect of fiscal policy on output exist.  While the general IS-LM model predicts an increase in government spending to have an expansionary effect on output, standard real business cycle (RBC) models predict the opposite depending on assumptions of non-Ricardian versus Ricardian consumers \parencite{gali2007understanding}.  It is important that policymakers understand the effects of a given policy as, for example, prolonged budget deficits can have significant negative effects on long-run economic outcomes through reduced national savings and higher interest rates \parencite{gale2003economic}.

We estimate the effect of government spending on the economy with the fiscal multiplier, which measures the change in output in response to a change in fiscal policy \parencite{spilimbergo2009fiscal}.  The theory of the Keynesian multiplier suggests the multiplier is greater than one, meaning every dollar change in fiscal policy causes more than a dollar change in output through households and businesses spending the additional money provided by the government \parencite{barro2011macroeconomic}.  Alternatively, the theory of crowding out suggests rational-acting households respond to increases in spending by saving more since they know fiscal forces will need to readjust later, suggesting a multiplier less than one \parencite{berge2021fiscal}.  Estimates of the multiplier inform policymakers on the effect of their decisions, so an accurate understanding of how the multiplier effects work is important \parencite{eyraud2013challenge}.  Biased estimates of the multiplier used to guide policy during financial crises have hampered economic recovery \parencites{blanchard2013growth}{blanchard2014learning}.  A higher multiplier suggests very different optimal policy decisions than a lower one, especially in times of economic crisis.  In this paper, we estimate a multiplier around one using historical data on the US economy, suggesting increases in government spending cause an almost equal increase in output.

Economists use a variety of tools to estimate the multiplier. Quantitative modeling approaches, which use mathematical equations calibrated to match real decision-making processes to model the behavior of economic agents, tend to estimate a multiplier between 0.5 and 1 \parencite{gechert2012fiscal}. Standard statistical approaches use regressions, instrumental variables, and likelihood estimation find a multiplier between 0.25 and 0.75 \parencite{gechert2012fiscal}. This paper uses a structural vector autoregression (VAR) to estimate the multiplier effect.

Originating from \textcite{sims1980macroeconomics}, structural VARs introduce structural restrictions to reduced-form VAR models which allow for a causal interpretation of the estimates.  This general framework has been applied to study postwar business cycle fluctuations, oil shocks, and the effect of monetary policy regime on output \parencites{hamilton1983oil}{hodrick1997postwar}{sims2006were}. Our paper is based on the framework in \textcite{blanchard2002empirical}, which presents a structural VAR approach to estimate the fiscal multiplier. 
