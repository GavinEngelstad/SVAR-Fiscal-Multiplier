 %importance of good fiscal policy

 %Regulation of economy is important (political economy stuff??)
Monetary and fiscal policy are the two main channels through which the US economy is regulated.  Monetary policy involves the central bank exercising power over the aggregate money supply to change interest rates in an effort to impact consumption decisions.  Fiscal policy describes the federal government's efforts to affect the economy through the two main mechanisms of public spending and taxes.  
% Competing effects necessitate an optimal policy mix 

%Need some transition to just talking about fiscal

%how the multiplier factors into this

Competing theories of the effect of fiscal policy on output exist.  While the general IS-LM model predicts an increase in government spending to have an expansionary effect on output, standard real business cycle (RBC) models predict the opposite depending on assumptions of non-Ricardian versus Ricardian consumers \parencite{gali2007understanding}.  It is important that policymakers understand the effects of a given policy as for example, prolonged budget deficits can have significant negative effects on long-run economic outcomes through reduced national savings and higher interest rates \parencite{gale2003economic}. %given another example...or a few more

We can estimate the effect of government spending on the economy with the fiscal multiplier.  
% The fiscal multiplier is \dots.
This value can be greater than or less than one as it represents the ratio of the change in output for a change in government expenditure or taxes.  Some theory suggests a Keynesian multiplier through which the effect of increased spending is amplified beyond the initial amount \parencite{barro2011macroeconomic}.  However, under certain conditions increased government spending can lead to crowding out, whereby the public spending actually drives down private spending, contrary to neoclassical theory \parencite{berge2021fiscal}.

Understanding multipliers informs understanding short-term effects of fiscal policy as it allows policymakers to estimate the effect of their decisions \parencite{eyraud2013challenge}.


%some data/graphs??


%This paper explores.... we find....
This paper explores the impact of fiscal policy on the United States economy.  We estimate the fiscal multiplier using structural vector autoregression (SVAR) and federal reserve data from 1960-2007, finding a negligible effect of government expenditure on GDP robust to several specifications.